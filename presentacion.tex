
\documentclass{beamer}
% Replace the \documentclass declaration above
% with the following two lines to typeset your
% lecture notes as a handout:
%\documentclass{article}
%\usepackage{beamerarticle}
\usepackage{graphicx,wrapfig,lipsum}
%\graphicspath{ {images_latex/} }

\useoutertheme{infolines}
\usetheme{Hannover}


\title{MyRoute}


\subtitle { \vspace{1cm} \color{black}Juan Germ\'an G\'omez G\'omez \\ \'Alvaro L\'opez Jim\'enez \\ Antonio Jos\'e Camarero Ortega \\ Rub\'en Mar\'in Asunci\'on \\ Rub\'en M\'ogica Garrido \\ Alejandro Ru\'iz Becerra \\ Valent\'in Pedrosa Campoy}

\date{18/01/2019}
% - Either use conference name or its abbreviation.
% - Not really informative to the audience, more for people (including
%   yourself) who are reading the slides online

\subject{Prácticas DGP}
% This is only inserted into the PDF information catalog. Can be left
% out.

% If you have a file called "university-logo-filename.xxx", where xxx
% is a graphic format that can be processed by latex or pdflatex,
% resp., then you can add a logo as follows:

% \pgfdeclareimage[height=0.5cm]{university-logo}{university-logo-filename}
% \logo{\pgfuseimage{university-logo}}

% Delete this, if you do not want the table of contents to pop up at
% the beginning of each subsection:
\AtBeginSubsection[]
{
  \begin{frame}<beamer>{}
    \tableofcontents[currentsection,currentsubsection]
  \end{frame}
}

% Let's get started
\begin{document}


\begin{frame}
  \titlepage 
\end{frame}

\begin{frame}{\'Indice}
  \tiny
 
  \tableofcontents
  % You might wish to add the option [pausesections]
\end{frame}

% Section and subsections will appear in the presentation overview
% and table of contents.
\section{Gesti\'on del Equipo}

\subsection{Organizacion del Equipo}

\begin{frame}{Titulo diapositiva}{Optional Subtitle}
  \begin{itemize}
  \item {
    My first point.
  }
  \item {
    My second point.
  }
  \end{itemize}
\end{frame}

\subsection{Comunicacion}

\begin{frame}{Titulo diapositiva}{Optional Subtitle}
  \begin{itemize}
  \item {
    My first point.
  }
  \item {
    My second point.
  }
  \end{itemize}
\end{frame}

\subsection{Seguimiento}

\section{Gesti\'on del Proyecto}

\subsection{Tareas realizadas en cada iteracion}

\subsection{Gestion de la calidad y Accesibilidad}

\section{Arquitectura y tecnologia}

\subsection{Arquitectura del sistema}

	\begin{frame}{Arquitectura del sistema}
		\begin{itemize}
			\item {
				Back-end : API REST.
				
			}
			\item {
				Front-end : SPA (Single Page Application).
			}
			\item {
				Base de datos : Relacional.
				%\begin{wrapfigure}{r}{0cm}
				%\includegraphics{images_latex/postgresql}
				%\end{wrapfigure} 
				\hspace{1cm}
				%\includegraphics[scale=0.1]{images_latex/postgresql}
				
			}
		\end{itemize}
		
		
		\begin{figure}[h]
    \centering
    \includegraphics[width=0.25\textwidth]{images_latex/api}

\end{figure}
		%\includegraphics[scale=0.1]{images_latex/api}
	
	\end{frame}

\subsection{Herramientas y tecnologia}

\section{Pliego Tecnico}

\subsection{Cumplimiento del Pliego}

\subsection{Valor Anadido}

\section{Retrospectiva}

\subsection{Analisis DAFO}

\subsection{Lecciones aprendidas}

\subsection{Autocritica y critica a la asignatura}

\subsection{Valoracion Final}

% You can reveal the parts of a slide one at a time
% with the \pause command:
%\begin{frame}{Second Slide Title}
%  \begin{itemize}
%  \item {
%    First item.
%    \pause % The slide will pause after showing the first item
%  }
%  \item {
%    Second item.
%  }
  % You can also specify when the content should appear
  % by using <n->:
%  \item<3-> {
%    Third item.
%  }
%  \item<4-> {
%    Fourth item.
%  }
  % or you can use the \uncover command to reveal general
  % content (not just \items):
%  \item<5-> {
%    Fifth item. \uncover<6->{Extra text in the fifth %item.}
%  }
%  \end{itemize}
%\end{frame}




\end{document}
